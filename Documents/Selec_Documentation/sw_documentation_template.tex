\documentclass[11pt]{article}

%-------------------------------------------
%use package
\usepackage[margin=1in]{geometry}
\usepackage{amsfonts, amsmath, amssymb}
\usepackage[none]{hyphenat}
\usepackage{graphicx}
\usepackage{float}
\usepackage{fancyhdr}
\usepackage{enumitem}
\usepackage[nottoc, notlot, notlof]{tocbibind}

\pagestyle{fancy}
\fancyhead{}
\fancyfoot{}
\fancyhead[L]{\uppercase{\textit{Interfacing Microcontroller}}}
\fancyhead[R]{\textit{Ashwini Kumar Gupta}}
\fancyfoot[C]{\thepage}

\renewcommand{\footrulewidth}{2pt}
\parindent 0ex
%\setlength{\parindent}{4em}
%\setlength{\parskip}{1em}
\renewcommand{\baselinestretch}{1.5}
\newcounter{imageCounter}
\refstepcounter{imageCounter}
%-------------------------------------------

\begin{document}
	
	\begin{titlepage}
		\begin{center}
			\vspace*{1cm}
			\Large{\textsc{Selec Controls}}\\
			%\Large{\textbf{Transformer Power Supply}}
			\vfill
			\line(1,0){400}\\[1mm]
			\Huge{\textsc{Interfacing Microcontroller}}\\ [3mm]
			\huge{\textbf{-HW Assignment 11-}}\\ [1mm]
			\line(1,0){400}\\
			\vfill 
			\large{By : Ashwini Kumar Gupta\\
			Dept. Of Electronics \& Telecommunication\\
			VES Institute of Technology, Chembur \\}
			\today\\
			
		\end{center}
	\end{titlepage}
	
	\tableofcontents
	\thispagestyle{empty}
	\clearpage
	
	\setcounter{page}{1}
	
	\section{Objective}
		To develop knowledge of selecting \& Interfacing Microcontroller as per
Application requirement.
		
	\section{Assignment Details}
	Select 'Renesas RL78 / G12 family ' Microcontroller suitable for following Application	
		\begin{enumerate}
			\item Interface 12VDC, 20mA Buzzer (use HA6 Circuit )
			\item Interface HONGFA Relay , 1Form A , 12 VdC coil supply , 10A Contact rating (use HA9 circuit)
			\item Sense Healthiness of Mains AC Power supply , Mains supply > 100VAC is healthy condition ( use HA8 circuit)
			\item Measure Sensor response , which O/P is in the range of 5mV to 100mV, amplified suitably for Single ended ADC of selected Microcontroller (use HA7 circuit)
			\item  Interface non isolated RS 485 port. (use TI make SN65HVD08)
		\end{enumerate}
		
	\section{Design Considerations}
		\begin{itemize}
			\item[Microcontroller Selection:]
				\begin{itemize}
			\item The microcontroller should be able to handle analog voltage from the op-amp circuit, i.e. inbuilt ADC.
			\item Should be able to understand TTl logic.
			\item Low power consuming
			\item should provide with at least 1 interrupt pin for RS485.
			\item min of 2KB of ROM.
			\item The Renesas RL78 / G12 family has a device R5F1026AASP  suitable for this application.
				\end{itemize}
		
			\item[Crystal and capacitors:]
				\begin{itemize}
					\item The for operating at 12MHz
					\item Crystal ECS-120-8-33B-CZM-TR
					\item the crystal cpacitors required will be,\\
						$$ C_{l1} = C_{l2} = 2*(C_l - C_p)$$
						From data sheet we get $C_l$ = 8 pf, $C_p =  2pf$
						$$ C_{l1} = C_{l2} = 12 pf$$
				\end{itemize}
				
			\item[Reset]
				\begin{itemize}
					\item Reset pin is active low.
					\item the Reset pin will be connected to 3.3 V through a 1K$ \Omega$ resistor and to GND through a push button.
					 
				\end{itemize}
			\item[Selection of Pins for I/P and O/P devices]
				\begin{itemize}
					\item[Buzzer]
						\begin{itemize}
							\item It is an Output device requiring 1 GPIO with low current requirement.
							\item attach Buzzer circuit to PIN 17.
							\item Pin 17 is a PCLBUZ0 i.e a programmable pin dedicated to buzzer.
						\end{itemize}
					\item[Relay]
						\begin{itemize}
							\item It is an Output device requiring 1 GPIO with low current requirement.
							\item attach Buzzer circuit to PIN 19.
						\end{itemize}
						
					\item[Opto Isolator circuit]
						\begin{itemize}
							\item It is an input circuit with TTL logic.
							\item only requires Digital input pin.
							\item assigning PIN 4/ PORT 40, since it is only input pin.
						\end{itemize}
					\item[Op-Amp Circuit]
						\begin{itemize}
							\item It is an input circuit with analog input.
							\item Requires ADC pin.
							\item Assigning pin 18 since it is an Analog Input pin (ANI3)
						\end{itemize}
					\item[RS485]
						\begin{itemize}
							\item Requires a Digital Tx and Rx pin
							\item Pin 15 is TxD0 for D (of RS485)
							\item Pin 16 is RxD0 for R (of RS485).
							\item Since RS485 is half duplex, we can control the RE and DE Pin from same interrupt pin of microcontroller.
							\item Interrupt Pin 6 i.e INTP0 for RE and DE 
						\end{itemize}
					
					\item[Default]
						\begin{itemize}
							\item Vss to GND
							\item Vdd to 3.3V
							\item set TTL analog voltage ref at $AV_{refm}$ and      
							$ AV_{refp}$
						\end{itemize}
				\end{itemize}
		\end{itemize}
		
	\section{Block Diagram}
		
\end{document}